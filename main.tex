\documentclass{article}
\usepackage[utf8]{inputenc}


\makeatletter

\usepackage{blindtext}  % dummy text
\usepackage[a4paper, total={6in, 9in}]{geometry}
\usepackage{amsmath}
\usepackage{amssymb}
\usepackage{amsfonts}
\usepackage{physics}
\usepackage{color}
\usepackage{xcolor}

\newcommand{\tS}{\text{t}}   % for terminal states
\newcommand{\tSi}{\tau}   % for terminal states in the subtasks
\newcommand{\cA}{\mathcal{A}}
\newcommand{\cB}{\mathcal{B}}
\newcommand{\cC}{\mathcal{C}}
\newcommand{\cD}{\mathcal{D}}
\newcommand{\cE}{\mathcal{E}}
\newcommand{\cJ}{\mathcal{J}}
\newcommand{\cL}{\mathcal{L}}
\newcommand{\cM}{\mathcal{M}}
\newcommand{\cP}{\mathcal{P}}
\newcommand{\cR}{\mathcal{R}}
\newcommand{\cS}{\mathcal{S}}
\newcommand{\cT}{\mathcal{T}}
\newcommand{\cU}{\mathcal{U}}
\newcommand{\cV}{\mathcal{V}}
\newcommand{\cX}{\mathcal{X}}
%mine bold math symbols
\newcommand{\bx}{\mathbf{x}}
\newcommand{\bA}{\mathbf{A}}
\newcommand{\param}{\mathbf{w}}
\newcommand{\bb}{\mathbf{b}}
\newcommand{\bI}{\mathbf{I}}


\newcommand{\EE}[1]{\mathbb{E}\left[#1\right]}
\newcommand{\EEc}[2]{\mathbb{E}\left[#1\;\middle\lvert\;#2\right]}
\newcommand{\pa}[1]{\left(#1\right)}

\newcommand{\diag}{\text{diag}}

%\newtheorem{theorem}{Theorem}[section]
%\newtheorem{lemma}[theorem]{Lemma}
%\newtheorem{definition}[theorem]{Definition}
%\newenvironment{proof}[1][Proof]{\begin{trivlist}
%item[\hskip \labelsep {\bfseries #1}]}{\end{trivlist}}

\title{Notes on Reinforcement Learning}
\author{Guillermo Infante}
\date{\today}



\begin{document}

\maketitle
\section{Intro}
According to (Frances et al., 2021), a general policy to solving the domain $clear(x)$ is
\begin{align}
    r_1 &: \{\neg c, H, n=0\} \implies \{c, \neg H\}, \\
    r_2 &: \{\neg c, \neg H, n=0 \} \implies \{c?, H, n \downarrow\}, \\
    r_3 &: \{\neg c, H, n=0\} \implies \{ \neg H \},
\end{align}

\noindent In this case $c$ stands for block $x$ being clear. $H$ and $n$ are define the same way as in 

\[
    V^*(s) = h^*(s) = 2 n(s) + H(s) \\   
\]

\noindent For $\mathcal{Q}_{clear}$ is
\begin{itemize}
    \item $\lvert \mathcal F \rvert = 532$
    \item $ k_\mathcal{F} = 8$
    \item Max. complexity for feature in the solution $ k^*= 4$
    \item Number of selected features is $\lvert \pi_\phi \rvert = 3$
\end{itemize} 

\end{document}